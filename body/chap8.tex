\chapter{表格示例}{Examples of Tables}
输入表格并且制作表格样式是论文中的复杂部分, 这里我给出一些常用表格示例, 这些表格
代码大部分依赖于 tabu 宏包.

\begin{table}
  \centering
  \caption{加宽表格}{Wider table}\label{tab:WiderTable}
  \tabulinesep=1.5mm
  \begin{tabu}spread 2cm{|X[c,m]|X[c,m]|X[c,m]|}
    \hline\rowfont[c]{\bfseries}
    操作系统   & 发行版   & 编辑器\\ \hline
    Windows    & MikTeX   & TeXnicCenter\par WinEdit \\ \hline
    Unix/Linux & TeX Live & Emacs \\ \hline
    Mac OS     & MacTeX   & TeXShop \\ \hline
  \end{tabu}
\end{table}

\begin{table}
  \centering
  \caption{表格行加高}{Increase height in every row}
  \label{tab:IncreaseHeightInEveryRow}
  \tabulinesep=1.5mm\extrarowsep=1mm
  \begin{tabu}spread 2cm{|X[c,m]|X[c,m]|X[c,m]|}
  \hline
  操作系统& 发行版& 编辑器\\ \hline
  Windows & MikTeX & TeXnicCenter\par WinEdt \\ \hline
  Unix/Linux & TeX Live & Emacs \\ \hline
  Mac OS & MacTeX & TeXShop \\ \hline
  \end{tabu}
\end{table}

\begin{table}
  \centering
  \caption{三线式表格}{Three lines table}
  \label{tab:ThreeLinesTable}
  \tabulinesep=1.5mm
  \begin{tabu}to 0.7\linewidth{X[c,m]X[c,m]X[c,m]}
    \tabucline[0.08em]-
    操作系统& 发行版& 编辑器\\
    \tabucline-
    Windows & MikTeX & TeXnicCenter\par WinEdt \\
    Unix/Linux & TeX Live & Emacs \\
    Mac OS & MacTeX & TeXShop \\
    \tabucline[0.08em]-
  \end{tabu}
\end{table}

在制作三线式表格时也可以使用 booktabs 宏包中的 \verb|\toprule|, \verb|\midrule|,
\verb|\bottomrule|. 或者使用 \verb|\heavyrulewidth| 代替 \verb|0.08em|, 其实在
booktabs 宏包中, \verb|\heavyrulewidth=0.08em|.

\begin{table}
  \centering
  \caption{横向合并单元格}{Multicolumn in a table}
  \label{tab:MulticolumnInATable}
  \tabulinesep=1.5mm
  \begin{tabu}to 0.7\linewidth{X[c,m]X[c,m]X[c,m]}
    \tabucline[0.08em]-
    \multicolumn{3}{c}{\LaTeX{} 简介}\\
    \tabucline-
    操作系统& 发行版& 编辑器\\
    \tabucline-
    Windows & MikTeX & TeXnicCenter\par WinEdt \\
    Unix/Linux & TeX Live & Emacs \\
    Mac OS & MacTeX & TeXShop \\
    \tabucline[0.08em]-
    \tabuphantomline
  \end{tabu}
\end{table}

\begin{table}
  \centering
  \caption{纵向合并单元格}{Multirow in a table}
  \label{tab:MultirowInATable}
  \tabulinesep=1.5mm
  \begin{tabu}to 0.7\linewidth{X[c,m]|X[c,m]|X[c,m]|X[c,m]}
    \tabucline[0.08em]-
    操作系统& 发行版 & 编辑器& 用户体验\\
    \tabucline-
    Windows & MikTeX & TeXnicCenter\par WinEdt & \multirow{3}{*}[0.5cm]{很好}\\
    Unix/Linux & TeX Live & Emacs & \\
    Mac OS & MacTeX & TeXShop & \\
    \tabucline[0.08em]-
  \end{tabu}
\end{table}
tabu 没有纵向合并单元格的命令, 所以还需要借助 multirow 宏包中的 \verb|\multirow|,
其中 \verb|[0.5cm]| 用来提升文本的内容以使其垂直居中.


\begin{center}
  \tabulinesep=1.5mm
  \begin{tabu}to 0.7\linewidth{|[1.5pt]X[c,m]|X[c,m]|X[c,m]|[1.5pt]}
    \tabucline[1.5pt]-
    操作系统& 发行版& 编辑器\\
    \tabucline-
    Windows & MikTeX & TeXnicCenter\par WinEdt \\
    Unix/Linux & TeX Live & Emacs \\
    Mac OS & MacTeX & TeXShop \\
    \tabucline[1.5pt]-
  \end{tabu}
\end{center}


\begin{center}
  \tabulinesep=1mm
  \tabcolsep=1mm
  \begin{tabu}{|X[c,m]|}
    \tabucline-
    \begin{tabu}{|X[c,m]|X[c,m]|X[c,m]|}
      \tabucline-
      操作系统& 发行版& 编辑器\\
      \tabucline-
      Windows & MikTeX & TeXnicCenter \\
      Unix/Linux & TeX Live & Emacs \\
      Mac OS & MacTeX & TeXShop \\
      \tabucline-
    \end{tabu}\\
    \tabucline-
  \end{tabu}
\end{center}

\begin{center}
  \tabulinesep=0.5mm
  \tabcolsep=0.5mm
  \begin{tabu}{|[2pt]X[c,m]|[2pt]}
    \tabucline[2pt]-
    \begin{tabu}{|X[c,m]|X[c,m]|X[c,m]|}
      \tabucline-
      操作系统& 发行版& 编辑器\\
      \tabucline-
      Windows & MikTeX & TeXnicCenter \\
      Unix/Linux & TeX Live & Emacs \\
      Mac OS & MacTeX & TeXShop \\
      \tabucline-
    \end{tabu}\\
    \tabucline[2pt]-
  \end{tabu}
\end{center}
