
\chapter{\LaTeX{}\index{\LaTeX} 应用情况}{The Application of \LaTeX}
我国已经有很多学术机构和高校校园网安装有 \TeX{}\index{\TeX} 或 \LaTeX{}\index{\LaTeX} 系统, 很多大专院校的教
师和学生、研究院所的科研人员以及出版社的编辑在使用 \LaTeX\index{\LaTeX}, 一些学术刊物开始接受使用
\LaTeX{}\index{\LaTeX} 排版的稿件。例如:《化学物理学报》已使用 C\TeX{} 排版英文期刊, 并鼓励作者
用 \LaTeX{}\index{\LaTeX} 排版投稿; 《计算数学》,《高能物理与核物理》,《电子与信息学报》,《应用数
学学报》等期刊的编辑部都要求作者提供 \LaTeX{}\index{\LaTeX} 源文件;《数学学报》和《工程数学学报》
更是明确指出定稿后作者必须提供 \LaTeX{}\index{\LaTeX} 源文件.

目前世界上许多权威学术机构都将 \LaTeX{}\index{\LaTeX} 排版格式作为标准的投稿文档格式. 例如: 国际
电子电气工程师协会, 美国工业和应用数学学会的各种期刊以及相关国际会议的论文都是将
\LaTeX{}\index{\LaTeX} 稿件列为首选; 美国数学学会将它所有出版物的稿件都要求用 \LaTeX{}\index{\LaTeX} 排版, 并提
供各种刊物的样式模板文件.

\LaTeX{}\index{\LaTeX} 作为一种专业的高品质排版系统, 已成为目前国际学术界最流行的排版系统, 很多
国际著名的出版机构都推荐或要求使用 \LaTeX\index{\LaTeX}. 例如: 荷兰爱思唯尔公司, 是世界著名的高
水准学术期刊出版商, 它出版的 1600 多种学术期刊中, 大部分都接受 \LaTeX{}\index{\LaTeX} 稿件, 有些
关于计算机、数学等方面的期刊还规定必须使用 \LaTeX{}\index{\LaTeX} 排版稿件, 并提供相应的 \LaTeX{}\index{\LaTeX}
样式模板和参考文献样式文件等; 德国施普林格公司是世界上著名的科技出版集团, 发行电子
图书并提供学术期刊检索服务, 目前共出版有 500 多种刊物, 其中大部分都可用 \LaTeX{}\index{\LaTeX} 投稿.
还有 Addison-Wesley, 牛津大学出版社等世界一流的出版社也都采用 \LaTeX{}\index{\LaTeX} 系统出版书籍
和期刊.
 
现在, 很多大型高科技企业也开始用 \LaTeX{}\index{\LaTeX} 排版订货合同、产品说明等, 因为有些商务文
件包含大量图形表格、数据指标、技术要求和相关标准引用, 其复杂和精细程度不亚于科技论
文.

毕竟 \LaTeX{}\index{\LaTeX} 是赠送给科技人员的写作工具, 它主要还是应用在科研报告和学术论文的排版,
用户大都是科研机构的研究人员和大专院校的师生. 虽然 \LaTeX{}\index{\LaTeX} 不是一天就能掌握的排版
系统, 但是一旦开始使用就会被它无穷的魅力所吸引. 随着科技的进步、网络时代的到来,
一定会有越来越多的科研工作者喜欢上 \LaTeX\index{\LaTeX}.

实际上, \LaTeX{}\index{\LaTeX} 已经逐渐成为科技文件发行和交流的主流出版标准.
